%!TEX TS-program = xelatex
%!TEX encoding = UTF-8 Unicode

%NB if you change paper size, change it in preamble too (where geometry is loaded)
\documentclass[11pt,a4paper]{extarticle}
% extarticle is like article but can handle 8pt, 9pt, 10pt, 11pt, 12pt, 14pt, 17pt, and 20pt text

\def \ititle {Is tracking knowledge a precondition of tracking belief?}
\def \isubtitle {Commentary on Jennifer Nagel's `Knowledge as a Mental State'}
\def \iauthor {Stephen A. Butterfill}
\def \iemail{s.butterfill@warwick.ac.uk}
%for anonymous submisison
%\def \iauthor {}
%\def \iemail{}
%\date{}

\input{$HOME/Documents/submissions/preamble_steve_paper}
\setromanfont[Mapping=tex-text]{Charis SIL} 

\begin{document}

\setlength\footnotesep{1em}

\bibliographystyle{newapa} %apalike

%these two lines are for anonymous submission --- they remove author and date
%but don't forget to remove defs above as well --- otherwise it will be in the metadata
%\author{}
%\date{}


\maketitle
%\tableofcontents

\begin{abstract}
\noindent
***


\end{abstract}


\section{***}
Arguments for the claim that knowledge is a mental state have been rejected by philosophers on several quite different grounds.
Fricker allows (possibly only for the sake of argument) that knowledge does, and is ordinarily taken, to explain action (p.\ 35) while rejecting the claim that knowledge is a mental state.\footnote{
See also FriThere is absolutely no tension between knowing's being a good explanatory state, and each instance of knowing being a conjunctive, hybrid phenomenon. 
}
%

What do human adults understand of knowledge?
Consider two claims about the nature of knowledge.



Nagel argues that they treat knowledge as explanatory of action, that they treat it as a state (rather than, say, as an ability), and that they treat it as a mental state (rather than, say, as a bodily state).
 

Is `the identification of knowledge as a mental state ...\ one of the central principles of our [human adults'] intuitive mindreading system?'




\section{Tracking vs.\ Representing}




In footnote 25 Nagel writes:
%
\begin{quote}
`By observing that chimpanzees have some capacity to recognize the state of knowledge, one need not thereby credit chimpanzees with any very sophisticated understanding of the nature of knowledge'
\end{quote}
%
I want to distinguish two issues.
One is whether 




\bibliography{$HOME/endnote/phd_biblio}

\end{document}