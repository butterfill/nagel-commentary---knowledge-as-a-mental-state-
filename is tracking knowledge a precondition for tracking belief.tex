%!TEX TS-program = xelatex
%!TEX encoding = UTF-8 Unicode

%NB if you change paper size, change it in preamble too (where geometry is loaded)
\documentclass[11pt,a4paper]{extarticle}
% extarticle is like article but can handle 8pt, 9pt, 10pt, 11pt, 12pt, 14pt, 17pt, and 20pt text

\def \ititle {What Does Knowledge Explain?}
\def \isubtitle {Commentary on Jennifer Nagel, `Knowledge as a Mental State'}
\def \iauthor {Stephen A. Butterfill}
\def \iemail{s.butterfill@warwick.ac.uk}
%for anonymous submisison
%\def \iauthor {}
%\def \iemail{}
%\date{}

\input{$HOME/Documents/submissions/preamble_steve_paper}
%\setromanfont[Mapping=tex-text]{Charis SIL} 

\begin{document}

\setlength\footnotesep{1em}

\bibliographystyle{newapa} %apalike

%these two lines are for anonymous submission --- they remove author and date
%but don't forget to remove defs above as well --- otherwise it will be in the metadata
%\author{}
%\date{}


\maketitle
%\tableofcontents

\begin{abstract}
\noindent
***


\end{abstract}


%`we [adult humans] intuitively attribute knowledge to others as a state which explains their actions' (p.\ 31) and, relatedly, that
%Where Nagel claims that `recognition of knowledge is prior to recognition of mere belief', 

\section{Preliminaries}
In defending the claim that knowledge is a mental state, Nagel aims to show that `the identification of knowledge as a mental state is one of the central principles of our intuitive mindreading systems' (p.\ 33).
Is it true that intuitive mindreading systems identify knowledge as a mental state?

Nagel argues for a positive answer
partly on the grounds that 
knowledge features in intuitive explanations of action.
This by itself is not sufficient grounds.
Even if knowledge features in intuitive explanations of action, 
it doesn't follow that
knowledge is identified as a mental state.
%But the inference from \emph{X features in intuitive explanations of action} to \emph{X is intuitively identified as a mental state} is not valid.
For things other than mental states, such as facts, can feature in intuitive explanations of action; and there is no reason to suppose that all such things are intuitively identified, incorrectly, as mental states.
Consider, for example, this explanation: Ayesha went inside because it was getting dark.
Note that this explanation could hold even if Ayesha neither knew nor believed that it was getting dark (changes in lighting can affect action independently of belief and knowledge).
As this indicates, appealing to facts allows us to explain actions which we could not explain by appeal to mental states only, and such explanations have greater generality in one dimension than comparable explanations involving mental states.
This is one reason, not decisive but significant, for holding that facts feature along with mental states in intuitive explanations of action.
Not everything which features in intuitive explanations of action should be identified as a mental state.

Are there any reasons to doubt  Nagel's claim that intuitive mindreading systems identify knowledge as a mental state?
One reason is that there seems to be no need for such systems to make an identification either way.
Among their roles are prediction and explanation of thought and action.
Fulfilling roles such as these surely involves identifying factors which predict or cause actions. 
But it doesn't seem to require taking a view on whether these factors are mental, nor even on whether they are states.%
\footnote{
\citet[p.\ 451]{Hyman:1999fk} argues that propositional knowledge is an ability `to act, to refrain from acting, to believe, desire or doubt for reasons that are facts.' 
}
If I were a mindreading system, I would want to remain neutral on which things are mental states.

So far I have only been skimming the surface of Nagel's argument.
A core aim of hers is to oppose the claim that knowledge is a hybrid state involving belief, truth and other factors.
Nagel sees this claim as a key reason for denying that knowledge is a mental state.
And she suggests that truths about the role of knowledge in intuitive mindreading are incompatible with the claim.
In outline, her argument (in nearly her own words, see pp.\ 4, 21, 33) is this:
%
\begin{enumerate}
\item The capacity to represent belief is not in place until after the capacity to represent knowledge is.
\end{enumerate}
This entails that:
\begin{enumerate}[resume]
\item Intuitive representation of knowledge cannot be `a composite involving intuitive representation of belief'.
\end{enumerate}
Which in turn is evidence that:
\begin{enumerate}[resume]
\item Knowledge is not `a composite of belief and non-mental factors'.
\end{enumerate}
%
It is not my intention to argue, contra Nagel, that knowledge is composite or that it is not a mental state.
For what it's worth,
I don't disagree with her on either claim.
But I do think there are several problems with the above line of argument.
Below I shall suggest that, on balance, the currently available evidence does not support (1).  
But first, is (2) really evidence for (3)?

To see why not, consider a parallel.
Imagine individuals who can represent coffee but not caffeine.
These individuals' intuitive representation of coffee cannot be a composite involving an intuitive representation of caffeine.
But, you are to imagine, coffee features in their explanations of action.
For instance, they explain variations in their own and others' performance by appeal to coffee consumption.
And in many cases appeal to coffee consumption allows them to give better (relative to their own ends, at least) explanations than they could give if they were to appeal to caffeine or other coffee components.
Clearly none of this is evidence that coffee is not a composite involving caffeine.
As this example indicates, 
non-composite representations of things which are in fact composite are 
not necessarily misrepresentations 
and not necessarily defective relative to the ends they serve.
%After all, few or no representations capture every aspect of the things they represent.
Even if knowledge were composite, having non-composite representations of it might still be advantageous in predicting and explaining action.%
\footnote{
\citet[p.\ 51]{fricker_2009} makes a different but related point: `There is absolutely no tension between knowing's being a good explanatory state, and each instance of knowing being a conjunctive, hybrid phenomenon.'
}
This is why even assuming `a non-skeptical attitude' (p.\ 33) to intuitive mindreading, (2) is not evidence for (3).%
\footnote{
Nagel also appears to suggest that anyone who, like \citet{Williamson:2000xz}, holds that belief should be explained in terms of knowledge 
has reason to hold that `the capacity to recognise belief depends on some prior mastery of the concept of knowledge' (p.\ 14).
This is not straightforward if, as Nagel allows elsewhere (in footnote 20), it is possible to recognise something without knowing everything about it and, in particular, without knowing every conceptual truth about it. 
}

We should also be cautious about the inference from (1) to (2).
Return to the coffee-but-not-caffiene-representing individuals.
Imagine a further stage of their development in which they acquire a capacity to represent caffeine.
Now we can no longer be sure that their representation of coffee is not a composite involving representations of caffeine.
After all, this further stage might involve a change in their representation of coffee.
Similarly, the fact (if it is a fact---see below) that humans acquire a capacity to represent knowledge before they can represent belief does not entail that their representation of knowledge is not \emph{eventually} a composite involving representation of belief.

So far I have argued that Nagel demands too much of intuitive mindreading systems.
As far as we know, their roles are bound up with the particular, with what Miss Kelly knows about Eilis and Tony and how this will shape her actions.
It is no part of their job to identify knowledge as a mental state,
nor to represent it in ways which reveal whether or not it is a composite involving belief.
While this is intended as an objection to some lines of argument in Nagel's paper, doesn't touch the core.




\section{Priorty}
In the previous section I examined some of the ways in which Nagel connects claims about mindreading to controversy over whether knowledge is a mental state.
In this section I want to set aside that controversy in order to focus just on mindreading.
Right at the core of her paper, Nagel argues that  `the concept of knowledge is in some sense prior to the concept of belief' (p.\ 21).%
\footnote{
On p.\ 14  Nagel expresses what I take to be a closely related thesis by saying that `the capacity to recognize belief depends on some prior mastery of the concept of knowledge'.
And near the start of the paper Nagel advertises the claim that `an ability to track what others would know seems to be the precondition, rather than the product, of an ability to track what they would believe' (p.\ 3).
It may be important to distinguish mastery of a concept of knowledge from abilities to track what others know.
}
I shall concentrate on the core argument and set aside issues Why think that the concept of knowledge is in any sense prior to the concept of belief?

Nagel is motivated in part by developmental evidence.
As she notes, earlier research shows that
children can reliably answer questions about knowledge and ignorance months or years before they can answer questions involving false belief \citep{hogrefe_ignorance_1986}.
So a three-year-old might be able reliably to report whether someone knows something and, relatedly, which of two or more people know it while systematically failing to correctly answer questions about what someone with a false belief will think, do or say \citep{Wellman:2001lz}.
Further, children's sensitivity to knowledge and ignorance appropriately guides a range of their actions, such as whether to rely on what someone says \citep{Robinson:1999sq,Robinson:2003bh} and whether to point to the location of an object \citep{Dunham:2000tv,Liszkowski:2008al}.
Nagel seems to interpret these findings as evidence that `the concept of knowledge [is] prior to the concept of belief' (p.\ 25). 
What Nagel doesn't mention, however, is that the picture becomes more complicated when we attempt to take a wider view and incorporate  recent research on infants' and also older children's abilities.

Infants are sensitive to others' false beliefs from some time around their seventh month or earlier \citep{kovacs_social_2010}.
From around their first birthday infants manifest sensitivity to belief in a variety of ways.
It is not just that infants look longer when an agent who apparently has a false belief acts as if she knew, which is evidence that such actions violate their expectations  \citep{Onishi:2005hm,Surian:2007hl}.
It is also that infants' eye movements anticipate actions based on false beliefs \citep{Southgate:2007js},
and that facts about others' false beliefs shape some communicative actions \citep{Knudsen:2011fk} and to some extent guide word-learning \citep{Carpenter:2002gc} as well as modulating how subjects help others \citep{Buttelmann:2009gy}.
Of course none of this establishes (at least not in any straightforward way) that typically developing humans do not acquire a concept of knowledge before they acquire a concept of belief.
After all, it is an open question whether infants' abilities are best explained by their having a concept of belief.
And if we do accept that these abilities are evidence that infants have a concept of belief, it remains possible that humans manifest sensitivity to knowledge earlier than to belief.
But taking a slightly wider view of developmental research does show, contra Nagel, that the available evidence does not  straightforwardly support the claim that capacities to ascribe knowledge come before  capacities to ascribe  belief.


***

Missing evidence at the other end:
Understanding of knowledge continues to develop: sources of knowledge, question-asking and [Early Understanding of the Division of Cognitive Labor Donna J. Lutz, Frank C. Keil].
At what point exactly do we have evidence that what children are representing is knowledge?
Shouldn't be too impressed by the fact that children can distinguish knowledgable from ignorant speakers (electricity).
[Incorporate footnote]%
\footnote{
Note that there is some debate about whether two- and three-year-olds' talk about and sensitivity to knowledge is best explained by supposing that they have a concept of knowledge.
It may be that children are sensitive to whether others are engaged or disengaged in an event \citep{ONeill:1996um}, and, separately, to whether others have a history of reliability or not.
This would ****

} 

***

[footnote goes somewhere]% 
\footnote{
While this research has recently attracted renewed interest, some of the credit should also go to earlier work. 
See \citet{Clements:1994cw,Garnham:2001jm,Garnham:2001ql,Ruffman:2001ng}
}


%

 
 

\section{Distinctions}

The answer may depend on what is involved in having a capacity  to recognize belief and a prior mastery of the concept of knowledge.%
\footnote{
Nagel also puts what I take to be the same thesis by saying that `an ability to track what others would know seems to be the precondition, rather than the product, of an ability to track what they would believe' (p.\ 3).
It may be important to distinguish mastery of a concept of knowledge from abilities to track what others know.
}
So before addressing this question directly it will be helpful to have some distinctions in place.
The first distinction is between theory of mind abilities and theory of mind cognition.
A theory of mind \textit{ability} is an ability that exists in part because exercising it brings benefits obtaining which depends on exploiting or influencing facts about others’ perception, knowledge, belief, desire or any other psychological states.  
Theory of mind \textit{cognition}  involves representing psychological states.
This distinction matters because not all theory of mind abilities involve theory of mind cognition.
To illustrate, some animals are able to modify their behaviour  in the presence of eyes directed at them \citep[e.g.][]{ernest-jones_effects_2011}.
It is plausible that this ability exists in part because it enables individuals to influence what others perceive.
But it doesn't follow, of course, that exercising this ability involves representing others' perceptions; for all we have said, the ability might depend on representing eyes and not perceptions.  



I shall interpret talk about \textit{tracking} mental states in terms of theory of mind abilities.
To track what another knows is to exercise theory of mind abilities concerning knowledge; this may but need not involve representing the other's knowledge.
By contrast, I suppose that \textit{recognising} and \textit{ascribing} knowledge does involve representing knowledge states.




Nagel's core claim concerns theory of mind cognition, not only theory of mind abilities. 
The issue concerns representations of knowledge and belief, not merely abilities to track them.%




Talk about tracking mental states is a natural 

***

\section{Intention}
I want to start by stepping back from this question in order to compare and contrast Nagel on knowledge with Bratman on intention.
Bratman aims to show that intention is not reducible to belief and desire,
and he aims to do this by identifying functional and normative roles for intention which cannot be played by belief or desire.
A theme  common to Bratman and Nagel, then, is that if someone were to refrain from ascribing intention or knowledge and confine herself to belief and desire only, her abilities to explain thought and action would be compromised.
Although Nagel's main argument on this point depends on an apparent contrast in explanatory generality, it may be that the explanatory power of knowledge depends on ***Hawthorne practical reasoning.



Bratman's position is consistent with the view that understanding intention is a more sophisticated achievement than understanding belief and desire.




\section{Introduction}
Nagel contrasts two views.
On the view she rejects, adult humans' `understanding of action is fundamentally dependent on belief attribution' and in ascribing knowledge states we are in some way drawing on an understanding of belief (p.\ 3).
Her own view is that `the capacity to recognize belief depends on some prior mastery of the concept of knowledge' (p.\ 14).%
\footnote{
Nagel also puts her view by saying that `an ability to track what others would know seems to be the precondition, rather than the product, of an ability to track what they would believe' (p.\ 3).
As explained below, it may be important to distinguish mastery of the concept of knowledge from abilities to track what others know.
}
Put roughly, the contrast concerns whether being able to think about knowledge depends on being able to think about belief or whether the converse dependence holds.
Of course these two positions may be consistent: perhaps there are dependencies running both ways.  





\section{***}
Arguments for the claim that knowledge is a mental state have been rejected by philosophers on several quite different grounds.
Fricker allows (possibly only for the sake of argument) that knowledge does, and is ordinarily taken, to explain action (p.\ 35) while rejecting the claim that knowledge is a mental state.\footnote{
See also FriThere is absolutely no tension between knowing's being a good explanatory state, and each instance of knowing being a conjunctive, hybrid phenomenon. 
}
%

What do human adults understand of knowledge?
Consider two claims about the nature of knowledge.



Nagel argues that they treat knowledge as explanatory of action, that they treat it as a state (rather than, say, as an ability), and that they treat it as a mental state (rather than, say, as a bodily state).
 

Is `the identification of knowledge as a mental state ...\ one of the central principles of our [human adults'] intuitive mindreading system?'




\section{Tracking vs.\ Representing}




In footnote 25 Nagel writes:
%
\begin{quote}
`By observing that chimpanzees have some capacity to recognize the state of knowledge, one need not thereby credit chimpanzees with any very sophisticated understanding of the nature of knowledge'
\end{quote}
%
I want to distinguish two issues.
One is whether 




\bibliography{$HOME/endnote/phd_biblio}

\end{document}